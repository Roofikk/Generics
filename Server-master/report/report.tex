%%% Для сборки выполнить 2 раза команду: pdflatex <имя файла>

\documentclass[a4paper,12pt]{article}

\usepackage{ucs}
\usepackage[utf8x]{inputenc}
\usepackage[russian]{babel}
%\usepackage{cmlgc}
\usepackage{graphicx}
\usepackage{listings}
\usepackage{xcolor}
%\usepackage{courier}

\makeatletter
\renewcommand\@biblabel[1]{#1.}
\makeatother

\newcommand{\myrule}[1]{\rule{#1}{0.4pt}}
\newcommand{\sign}[2][~]{{\small\myrule{#2}\\[-0.7em]\makebox[#2]{\it #1}}}

% Поля
\usepackage[top=20mm, left=30mm, right=10mm, bottom=20mm, nohead]{geometry}
\usepackage{indentfirst}

% Межстрочный интервал
\renewcommand{\baselinestretch}{1.50}


\begin{document}

%%%%%%%%%%%%%%%%%%%%%%%%%%%%%%%
%%%                         %%%
%%% Начало титульного листа %%%

\thispagestyle{empty}
\begin{center}


\renewcommand{\baselinestretch}{1}
{\large
{\sc Петрозаводский государственный университет\\
Институт математики и информационных технологий\\
	Кафедра \textcolor{red}{<наименование кафедры руководителя>}
}
}

\end{center}


\begin{center}
%%%%%%%%%%%%%%%%%%%%%%%%%
%
% Раскомментируйте (уберите знак процента в начале строки)
% для одной из строк типа направления  - бакалавриат/
% магистратура и для одной из
% строк Вашего направление подготовки
%
% Направление подготовки бакалавриата \\
% 01.03.02 Прикладная математика и информатика \\
% 09.03.02 - Информационные системы и технологии \\
% 09.03.04 - Программная инженерия \\
% Направление подготовки магистратуры \\
% 01.04.02 - Прикладная математика и информатика \\
% 09.04.02 - Информационные системы и  технологии \\
%
% 
%%%%%%%%%%%%%%%%%%%%%%%%%
	\textcolor{red}{<Ваши тип и направление подготовки>} 
\end{center}

\vfill

\begin{center}
{\normalsize Отчет о практике по научно-исследовательской работе} \\
	\textcolor{red}{<только для студентов направления 09.03.04:
	Отчет по научно-исследовательской практике>}

\medskip

%%% Название работы %%%
	{\Large \sc \textcolor{red}{<наименование темы работы>}} \\
	(промежуточный)
\end{center}

\medskip

\begin{flushright}
\parbox{11cm}{%
\renewcommand{\baselinestretch}{1.2}
\normalsize
	Выполнил:\\
%%% ФИО студента %%%
студент \textcolor{red}{<№ курса>} курса группы \textcolor{red}{<№ группы>}
\begin{flushright}
	\textcolor{red}{<И. О. Фамилия>} \sign[подпись]{4cm}
\end{flushright}
%%%%%%%%%%%%%%%%%%%%%%%%%
% девушкам применять "Выполнила" и "студентка"
%%%%%%%%%%%%%%%%%%%%%%%%%



Место прохождения практики: \\ \textcolor{red}{наименование кафедры или предприятия}\\

Период прохождения практики: \\
% Раскмментируйте строку с нужным Вам периодом прохождения.
%	\textcolor{red}{Для 4 к. бакалавриата и 6 к. магистратуры:} \\
% Период прохождения практики: \\ 02.09.19-15.12.19 \\
%	\textcolor{red}{Для остальных курсов:} \\
% Период прохождения практики: \\ 02.09.19-22.12.19 \\

% Если руководителей два - то раскомментровать строку про второго руководителя и применть "Руководители:"

Руководитель:\\
%%% степень, звание ФИО научного руководителя %%%
% Первый руководитель 
\textcolor{red}{И. О. Фамилия, ученая степень, ученое звание} \\
\begin{flushright}
\sign[подпись]{4cm}
\end{flushright}


% Второй руководитель 
% \textcolor{red}{И. О. Фамилия, ученая степень, ученое звание} \\
% \begin{flushright}
% \sign[подпись]{4cm}
% \end{flushright}



Итоговая оценка
\begin{flushright}
  \sign[оценка]{4cm}
\end{flushright}
}
\end{flushright}

\vfill

\begin{center}
\large
    Петрозаводск --- \textcolor{red}{<текущий год — 4 цифры>}
\end{center}

%%% Конец титульного листа  %%%
%%%                         %%%
%%%%%%%%%%%%%%%%%%%%%%%%%%%%%%%

%%%%%%%%%%%%%%%%%%%%%%%%%%%%%%%%
%%%                          %%%
%%% Содержание               %%%

\newpage

\tableofcontents

%%% Содержание              %%%
%%%                         %%%
%%%%%%%%%%%%%%%%%%%%%%%%%%%%%%%


%%%%%%%%%%%%%%%%%%%%%%%%%%%%%%%%
%%%                          %%%
%%% Введение                 %%%

%%% В введении Вы должны описать предметную область, с которой связана %%%
%%% Ваша работа, показать её актуальность, вкратце определить цель     %%%
%%% исследования/разработки					       %%%


\newpage
\section*{Введение}
\addcontentsline{toc}{section}{Введение}


Цель практики: \textcolor{red}{<определяет руководитель>} \\

Задачи практики: \textcolor{red}{<определяет руководитель>} \\

\textcolor{red}{<Текст введения>} \\

\textcolor{red}{<Текст отчета>} \\

%%%                          %%%
%%%%%%%%%%%%%%%%%%%%%%%%%%%%%%%%

\end{document}
